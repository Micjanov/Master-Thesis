\chapter{Utilizing hyperdimensional vectors in combination with neural networks for residue-level predictions}
As discussed in section~\ref{sec:nlp}, state-of-the-art protein models primarily rely on transformer-based architectures to learn representations of protein sequences and amino acids. These models have demonstrated their effectiveness in various applications, such as protein function prediction and protein-protein interaction prediction, by capturing complex patterns in amino acid sequences. However, these models are computationally intensive and may be less suitable for certain tasks or scenarios where computational resources are limited.

In section~\ref{sec:trans}, we explored the idea of encoding information from a residue's neighborhood into a hyperdimensional vector that represents the residue and its surroundings. This approach allows us to capture local sequence information and can serve as an effective encoding scheme for amino acids. To evaluate the effectiveness of this encoding method, we will employ it as an input for several perceptron-based models.

Perceptron-based models, a type of artificial neural network model, have been employed for various machine learning tasks over the years, including classification and regression problems. The perceptron, one of the simplest and earliest types of neural networks, was proposed by Frank Rosenblatt in 1958~\cite{perceptron}. A perceptron consists of a single artificial neuron that receives multiple input features and computes a weighted sum of these inputs. This weighted sum is then passed through an activation function, which determines the output of the perceptron. The learning process involves adjusting the weights of the input features iteratively, using a rule that minimizes prediction errors on the training data. The perceptron learning algorithm starts with an initial set of weights, typically set to zero or small random values. For each instance in the training data, the perceptron computes the output, compares it with the true label, and updates the weights accordingly. The weight update rule is based on the difference between the predicted output and the true label, multiplied by the learning rate and the input feature value. Single-layer perceptrons consist of an input layer and an output layer of perceptrons, allowing for multiclass classifications. However, they can only solve linearly separable problems, limiting their applicability to more complex tasks.

As single-layer perceptrons can only solve linearly separable problems, researchers developed more complex neural network architectures, such as multi-layer perceptrons (MLP), which consist of multiple layers of artificial neurons connected in a feedforward manner. MLPs can learn non-linear patterns in the data by using non-linear activation functions (e.g., sigmoid, tanh, ReLU) and employing backpropagation for weight updates~\cite{mlp}. This enhanced ability to model non-linear relationships allows MLPs to tackle a wider range of problems.

By combining the residue neighborhood encoding method with perceptron-based models, we could develop more computationally efficient alternatives to state-of-the-art transformer-based models. These models may provide valuable insights into protein sequences and amino acids while being more suitable for scenarios with limited computational resources.

\section{Methods}
We used the training dataset compiled by Klausen \textit{et al.} for their model protein language model Netsurf 2.0~\cite{netsurf} as our training dataset. It contains 10,337 protein sequences obtained from the Protein Data Bank (PDB)~\cite{pdb}. Each residue in these sequences is annotated with its 3- and 8-class secondary structure classification, providing a comprehensive description of the protein's structural properties. The residues have been encoded using our neighborhood-encoder as described in section~\ref{sec:trans} with $k=25$ and $k=100$ using random hyperdimensional vectors for each amino acid. Due to computational memory constraints, we randomly selected 60 \% of the sequences from the original dataset for training and testing, ensuring a representative sample of the data.

As for the perceptron-based model, to decrease the computational time, we opted to build, train and evaluate the model in Python due to its support for computations on GPUs on our systems. We employed PyTorch-Lightning v1.8.4 for model construction and training and CUDA v11.7.0 on a high-performance computing (HPC) cluster equipped with an NVIDIA V100 GPU. The model comprised fully-connected layers with an input layer size equal to the dimension of the hyperdimensional vectors (10,000 in this case). Depending on the secondary structure classification desired, the output layer size was set to either 3 or 8. We used a batch size of 128 during training. Between each layer, we incorporated a ReLU (Rectified Linear Unit) activation function to introduce non-linearity into the model. The training loss was monitored using cross-entropy loss, which is well-suited for multi-class classification problems. To optimize the model, we employed the widely-used ADAM optimizer. We evaluated various configurations and hyperparameters to determine the best-performing but still computationally feasible approach:

\begin{outline}
    \1 SLP
      \2 Learning rate: 0.03
      \2 Epochs: 100
    \1 MLP with one hidden layer
      \2 Learning rate: 0.003
      \2 Epochs: 200
      \2 Hidden layer size: 500
    \1 MLP with 10 hidden layers
      \2 Learning rate: 0.0001
      \2 Epochs: 200
      \2 Hidden layer sizes: 8000, 5000, 2000, 1000, 800, 500, 200, 100, 50, 20
\end{outline}

As test datasets, we opted to use the sequences CB513~\cite{cb513} and CASP12~\cite{casp12} since they are easily attainable and commonly used by researchers as benchmark datasets for protein language models. The MLP model with 10 hidden layers was trained using 60 \% of the training data and then evaluated using these test datasets.
\section{Results}
