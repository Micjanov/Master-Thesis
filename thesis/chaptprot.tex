\section{Protein research}
\subsection{Protein structures}
Proteins are an essential part of molecular biology and are responsible for a wide variety of functions. They are part of the important biological macromolecules that make up life, hence a lot of effort has gone towards trying to understand the functions of protein and disruptions in its mechanisms that lead to many kinds of diseases. To start, proteins are composed of a linear chain of amino acids (AA) with a length ranging from 50 to tens of thousands of AAs, all connected by peptide bonds into a polypeptide. This is also referred to as the \textit{primary structure} of a protein as mentioned earlier \cite{primstruct}. A sequence of amino acids is mostly determined by the genetic code without considering post-translational and post-transcriptional modifications etc. In the genetic code of all living organisms, there are 20 different kinds of amino acids coded in that make up the 'language' of proteins. This sequence of amino acids does not occur as a mere linear chain of peptides, however. A protein consists of much more intricasies. The polypeptide can also form locally folded structures due to chemical interactions within the backbone (the polypeptide chain without the R-group), referred to as the \textit{secondary structure} of a protein. The most well-known and common types of these are $\alpha$-helices and $\beta$-sheets. The overall three-dimensional structure of a protein is referred to as the \textit{tertiary structure}.
%In spite of that, for a large fraction of the approximately 20000 human proteins, the structures and functions remain still unknown.
%The current state-of-the-art methods for the identification of protein sequences are \textit{de novo sequencing} algorithms applied to tandem mass spectrometry data.\cite{protseq}

The most common way to determine the 3D structure of a protein has remained to be X-ray crystallography for more than half a century \cite{xray}, with cyro-electron microscopy now catching up rapidly.\cite{cyroem} However, these kinds of laboratory approaches for structure determination of proteins are not simple, expensive and in some cases not possible for the protein in question whilst sequence determination is relatively much easier to perform. For this reason, the number of verified three-dimensional structures has not kept up with the explosive growth in sequence information. On top of that, structure prediction is highly in demand for researchers in applications such as drug design. Therefore, a lot of effort has gone into computational methods for structure and function predictions from protein sequences. While a protein's structure and function are dynamic and dependent on its surroundings such as the cellular state and other proteins and molecules, it is still defined by its underlying sequence. This means that a lot of the 3D-structural and functional information of a protein should be retrievable from its amino acid sequence \cite{structure}, but our computational modeling abilities continue to be challenged by the complexity of the sequence-structure-function relationship in proteins. A lot of methods have been developed to tackle this problem, until recently, these methods were mainly based on statistical sequence models and physics-based structural simulations.

Physics-based approaches such as Rosetta \cite{rosetta} solves this problem by using atom energy functions and minimizing the total free energy of the system. These kind of methods had much succes, but also assume simplified energy models, are extremely computationally intensive and have a limited accuracy.\cite{review}

and can help in the prediction of secondary and tertiary 3-dimensional structures. To cope with the number of recorded protein sequences rising exponentially, far more compute-wise efficient methods based on multiple sequence alignments had to be developed like PSI-BLAST \cite{psiblast}, HHblits \cite{hhblits3} and MMseqs \cite{mmseqs2}. However, these methods might not be able to keep up with the ever-increasing number of protein sequences stored in databases.

\subsection{State-of-the-art protein language modeling}
It is intuitive to represent a protein as a sequence of letters with each letter corresponding to an amino acid. Likewise to natural languages, we can find common elements between naturally evolved proteins. These motifs and domains are essential to many biological processes and can easily be represented as words, phrases and sentences of amino acids.