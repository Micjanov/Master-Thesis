\section{Protein biology}
\subsection{Protein structure levels}
Proteins are an essential part of molecular biology and are responsible for almost all cellular functions. They are part of the important biological macromolecules that make up life, hence a lot of effort has gone towards trying to understand the functions of protein and disruptions in its mechanisms that lead to many kinds of diseases. Proteins are composed of a linear chain of amino acids (AA) with a length ranging from 50 to tens of thousands of AAs, all connected by peptide bonds into a polypeptide. This is also referred to as the \textit{primary structure} of a protein as mentioned earlier~\cite{primstruct}. A sequence of amino acids is mostly determined by the genetic code without considering post-translational and post-transcriptional modifications etc. In the genetic code of all living organisms, there are 20 different kinds of amino acids coded in that make up the 'language' of proteins. Each amino acid has the same backbone but differs by the chemical properties of its side chain, also known as the R-group. This sequence of amino acids does not occur as a mere linear chain of peptides and consists of much more intricacies, however. The polypeptide can form locally folded structures due to chemical interactions within the backbone (the polypeptide chain without the R-group), referred to as the \textit{secondary structure} of a protein. $\alpha$-helices and $\beta$-sheets are the most well-known and common examples of these structures. The overall three-dimensional structure that forms out of these structures is referred to as the \textit{tertiary structure}. These are formed due to interactions between the R-groups and are much harder to classify due to the quasi-infinite amount of different combinations that can occur in an amino acid sequence. And lastly, a protein can also be made up of multiple polypeptide chains, referred to as its subunits. These subunits together form the \textit{quaternary structure} of a protein.

The sought-after biochemical and cellular functions of a protein emerge from a combination of all of these structures. While a protein's 3D structure and function are dynamic and dependent on its surroundings such as the cellular state and other proteins and molecules, it is still defined by its underlying sequence. This means that a lot of the 3D-structural and functional information of a protein should be retrievable from its amino acid sequence~\cite{structure}. Understanding how a protein's sequence translates to its structure and function, otherwise known as the 'protein folding problem', is the central problem of protein biology and is crucial for understanding disease mechanisms and designing proteins and drugs for therapeutic and bioengineering applications. Therefore, a lot of effort has gone into computational methods for structure and function predictions from protein sequences but this sequence-structure-function relationship continues to challenge bioinformaticians (insert visual aid). Further in this chapter, we will discuss traditional and state-of-the-art bioinformatics methods to obtain more knowledge in this field.

\subsection{Computational methods for protein research}
As mentioned earlier, the Edman degradation method was mostly used before to determine the primary structure of a protein. The current state-of-the-art methods for the identification of protein sequences are \textit{de novo sequencing} algorithms applied to tandem mass spectrometry data~\cite{protseq} and allow simultaneous sequencing of thousands of proteins per given sample. Plenty of valuable biological and evolutionary information is already retrievable from just the primary structure \textit{via} traditional tools such as BLAST and MSA as discussed earlier and could provide more information for the prediction of secondary and tertiary structures. To cope with the number of recorded protein sequences rising exponentially, far more compute-efficient methods based on multiple sequence alignments had to be developed like PSI-BLAST~\cite{psiblast}, HHblits~\cite{hhblits3} and MMseqs~\cite{mmseqs2}. However, these methods might not be able to keep up with the ever-increasing number of protein sequences stored in databases.

At the other end of the spectrum, the most common way to determine the 3D structure of a protein has remained to be X-ray crystallography for more than half a century~\cite{xray}, with cyro-electron microscopy now catching up rapidly~\cite{cyroem}. However, these kinds of laboratory approaches for structure determination of proteins are complex, expensive and in some cases not possible for the protein in question whilst sequence determination is relatively much easier to perform. Because of that, the structures and functions for a large fraction of the approximately 20000 known human proteins remain unknown. The number of verified three-dimensional structures in protein databases consequently has not kept up with the explosive growth in sequence information, further increasing the demand for computational structure/function prediction models.

A lot of methods have been developed to tackle this problem. Until recently, these methods were mainly based on statistical sequence models or physics-based structural simulations. \textit{Ab initio} physics-based approaches such as ROSETTA~\cite{rosetta} solve this problem by searching the protein's conformational space using atom energy functions and minimizing the total free energy of the system. ROSETTA has shown to be effective at predicting unknown structures and has been widely used for varying applications, but also assumes simplified energy models, is extremely computationally intensive and has limited accuracies~\cite{review}. Statistical sequence modeling of a set of related proteins, on the other hand, has proven to be very useful for discovering evolutionary constraints, homology searches and predicting residue-residue contacts. Improvement of these models has mainly been data-driven; exploiting databases to build large deep learning systems which culminated in the recent success of DeepMind's AlphaFold2~\cite{alphafold2} at the Critical Assessment of protein Structure Prediction (CASP) 14. This succes of AlphaFold continued into more recently CASP 15~\cite{casp15}. Even though, DeepMind did not participate, the most successful participants integrated AlphaFold into their methods. However, all of these models are supervised methods that require labels. Labeled protein structure data essentially means retrieving the 3D coordinates of every atom in a protein which is very labor-intensive and time-consuming. Also, such kind of models would likely perform poorly when working with completely unrelated proteins since the model won't be trained on this kind of data. 

One underlying theme of these computational methods for protein structure prediction is being able to translate the protein 'language' into numerical representations which computers can learn from. This is now known as the field of protein language modeling which is more thoroughly discussed in section \ref{nlp}.