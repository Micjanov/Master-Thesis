\section{Goals and objectives}
After a historical overview of bioinformatics and in particular protein research, we raised the issues in the current landscape of protein language modeling and its need for more computationally and data-efficient methods. In response to this need, we took inspiration from hyperdimensional computing. This computing paradigm, initially thought out by P. Kanerva~\cite{Kanerva2009}, is inspired by the human brain's learning and adaptability capacities and provides an ultra-efficient and robust learning framework. The principal aim of this study is to explore the potential of hyperdimensional computing in protein sequence research, particularly focusing on the encoding of amino acids and protein sequences, evaluating their effectiveness in prediction tasks and understanding the performance and challenges along the way.

In the next chapter, we aim to provide an introduction to hyperdimensional computing, allowing the reader to gain an understanding of its underlying principles, operations and its potential for applications in the realm of bioinformatics and protein research by demonstrating its operations and applying them in small-scale examples.

In Chapter 3, the objective is to investigate and formulate strategies for encoding amino acids and protein sequences whilst leveraging the potential of hyperdimensional computing.

In the chapter thereafter, we aim to apply the developed encoding methods to a protein sequence database, PhaLP, a specialized resource for phage lytic proteins developed by Criel~\textit{et al.}~\cite{phalp}. Additionally, we plan to utilize the encoded sequences and the data of PhaLP as input information for classification tasks, experimenting with a variety of learning methods, whether or not based on hyperdimensional computing, to evaluate the effectiveness of our hyperdimensional embeddings in prediction tasks.

Lastly, in Chapter 5, the objective is to utilize hyperdimensional and context-aware amino acid representations as input information for amino acid-level prediction tasks to discover potential avenues for designing computationally efficient alternatives to state-of-the-art protein language models.