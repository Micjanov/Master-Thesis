\chapter{Abstract}
This dissertation explores the implementation and application of hyperdimensional computing for protein sequence analysis. We discussed the advancements of state-of-the-art protein language models in protein structure and function predictions while raising the need for more computationally and data-efficient methods. As hyperdimensional computing was proposed as a promising avenue, its underlying principles and mathematical operations were illustrated to demonstrate the potential of hyperdimensional computing in bioinformatics research. We researched and developed several methods to encode amino acids into hyperdimensional vectors. Of these, projecting embeddings containing biological information into hyperdimensional space has proven itself to be useful in subsequent analyses and prediction tasks. Utilizing the PhaLP database~\cite{phalp}, a continuously updated database of phage lytic proteins, we applied these amino acid encoding methods to develop techniques for protein sequence encodings. We demonstrate the capability of these methods to capture essential protein sequence information in hyperdimensional vectors, proving their usefulness in prediction tasks. In our classification tasks, we show that hyperdimensional computing-based learning methods displayed competitive performance when compared to established machine learning methods such as random forest and XGBoost. In addition, we examined perceptron-based models for context-aware protein residue learning, utilizing neighborhood-encoded hyperdimensional vectors. Although this did not outperform current state-of-the-art models, it contributed valuable insights into the challenges faced when implementing efficient models for such tasks within the hyperdimensional computing framework. Finally, we acknowledge the need for continued research in refining our encoding algorithms, exploring alternative model architectures, and extending the scope of tasks and datasets. Despite mixed results, our findings lay a solid foundation for further investigation into hyperdimensional computing's potential in protein sequence research and bioinformatics.