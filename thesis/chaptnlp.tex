\section{State-of-the-art protein language modeling}
It is intuitive to represent a protein as a sequence of letters with each letter corresponding to an amino acid. As with natural languages, we can find common elements between naturally evolved proteins. Noticeable patterns reoccurring in multiple (related) protein sequences are highly likely to be biologically relevant. These motifs and domains are essential to many biological processes and can easily be represented as words, phrases or sentences of amino acids in a language model perspective. This is why researchers are taking inspiration from the recent successes of natural language processing (NLP) and applying this to a biological context. NLP is a branch of artificial intelligence (AI) concerning itself with creating the ability for computers to learn and understand human languages by using statistical, machine learning and in recent years deep learning models. Common tasks in NLP which have been extended into protein sequence research include part-of-speech tagging, named entity recognition and natural language generation (token prediction).

As with protein modeling, applying labels to millions of natural language containing web pages, articles, journals etc.\ is a labor-intensive procedure and thus state-of-the-art NLP models use a form of \textit{self-supervised learning}, a form of unsupervised learning in which the context of the text is learned to fill in missing words, predict the next word in a sentence etc. during the training. Well-known NLP methods of this kind include bi-directional long-short term recurrent neural networks (biLSTMs) such as ELMo\cite{elmo} and more recently transformers such as Google's BERT\cite{bert} and OpenAI's GPT-3\cite{gpt3}. Despite the simplicity of these tasks, it is found to develop interesting capabilities as the scale of the model increases together with very little training on a specific task, now mostly referred to as \textit{few-shot learning}.

These deep-learning methods also show promise in the field of protein biology with the most notable latest projects being TAPE\cite{tape}, ProtTrans\cite{prottrans} and Meta AI's ESM-2\cite{esm2}, one of the most recent and largest protein language models to ever have been developed at the time of writing. They show much success in capturing evolutionary information and structure prediction tasks but at the expense of a lot of compute power anf