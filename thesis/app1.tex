\chapter{Define basic operations of hyperdimensional computing}
All provided code is written in the programming language Julia unless noted otherwise.
\section{Binary hypervectors}
\begin{minted}[breaklines, escapeinside=||,mathescape=true]{Julia}
using Random
"""
Construct a binary vector. By default 10000 elements long.
"""
bitHDV(N::Int=10000) = bitrand(N)
"""
Bundles binary hyperdimensional vectors based on the element-wise majority rule.
"""
function bitadd(vectors::BitVector ...)
    v = reduce(.+, vectors)
    n = length(vectors) / 2
    x = [i > n ? 1 : i < n ? 0 : rand(0:1) for i in v]
    return convert(BitVector, x)
end
"""
Binds binary hyperdimensional vectors based on an element-wise XOR gate.
"""
bitbind(vectors::BitVector ...) =  reduce(.|$\veebar$|, vectors)
"""
Permutes a binary hyperdimensional vector by an adjustable circular shift.
"""
bitperm(vector::BitVector, k::Int=1) = circshift(vector, k)
"""
Calculates the Hamming distance between two binary vectors.
"""
hamming(x::BitVector, y::BitVector) = sum(x .!= y)/length(x)
\end{minted}
\section{Bipolar hypervectors}
\begin{minted}[breaklines]{Julia}
using Random
using LinearAlgebra
"""
Construct a bipolar vector. By default 10000 elements long.
"""
hdv(N::Int=10000) = rand((-1,1), 1, N)
"""
Bundles bipolar hyperdimensional vectors.
"""
add(vectors::Vector{Int}...) = reduce(.+, vectors) .|> sign
"""
Binds bipolar hyperdimensional vectors.
"""
multiply(vectors::Vector{Int}...) = reduce(.*, vectors)
"""
Permutes a bipolar hyperdimensional vector by an adjustable circular shift.
"""
perm(vector::Vector, k::Int=1) = circshift(vector, (0, k))
"""
Calculates the cosine similarity between two bipolar vectors.
"""
cosine(x::Vector{Int}, y::Vector{Int}) = dot(x, y) / (norm(x) * norm(y))
\end{minted}
\chapter{Examples of hyperdimensional computing on datasets}
\section{Simple example}
\begin{minted}[breaklines]{Julia}
v = Vector()
for i in 1:10000
    a, b, c, x, y, z = bitHDV(), bitHDV(), bitHDV(), bitHDV(), bitHDV(), bitHDV()
    d = bitadd(bitbind(a, x), bitbind(b, y), bitbind(c, z))
    ap = bitbind(x, d)
    score = hamming(ap, a)
    append!(v, score)
end

using Plots
let
	p = histogram(v, xlims=(0,1), label="", xlabel="Hamming distance")
end
\end{minted}
\section{Zoo example}
\subsection{Data preprocessing}
\begin{minted}[breaklines]{Julia}
using DataFrames, CSV
data = CSV.read("zoodata/zoo.csv", DataFrame)
classls = data.class_type

mammal_idx = [i for i in 1:101 if data.class_type[i] == 1]
bird_idx = [i for i in 1:101 if data.class_type[i] == 2]
reptile_idx = [i for i in 1:101 if data.class_type[i] == 3]
fish_idx = [i for i in 1:101 if data.class_type[i] == 4]
amphibian_idx = [i for i in 1:101 if data.class_type[i] == 5]
bug_idx = [i for i in 1:101 if data.class_type[i] == 6]
invertebrate_idx = [i for i in 1:101 if data.class_type[i] == 7]

indices = [mammal_idx, bird_idx, reptile_idx, fish_idx, amphibian_idx, bug_idx, invertebrate_idx]

legs0 = [i == 0 ? 1 : 0 for i in data.legs]
legs2 = [i == 2 ? 1 : 0 for i in data.legs]
legs4 = [i == 4 ? 1 : 0 for i in data.legs]
legs5 = [i == 5 ? 1 : 0 for i in data.legs]
legs6 = [i == 6 ? 1 : 0 for i in data.legs]
legs8 = [i == 8 ? 1 : 0 for i in data.legs]

select!(data, Not(:legs))

data[!, :legs0] = legs0
data[!, :legs2] = legs2
data[!, :legs4] = legs4
data[!, :legs5] = legs5
data[!, :legs6] = legs6
data[!, :legs8] = legs8
\end{minted}
\subsection{Assigning HDVs to features}
\begin{minted}[breaklines]{Julia}
"""
Creates a list of binary hyperdimensional vectors representing an interval of numbers by constructing a random HDV representing the lower bound of the interval and replacing a fraction of the vector with random bits.
"""
function range_hdvs(steps)
	k = length(steps) - 1
	V = [bitHDV() for i in 1:k+1]
	for i in 2:k+1
		for j in 1:10000
			V[i][j] = rand(0:1)
		end
	end
	return V
end

legs_steps = 0:1:8
legs_hdvs = range_hdvs(legs_steps)

feature_hdv=[bitHDV() for i in 1:15]
append!(feature_hdv, [legs_hdvs[1], legs_hdvs[3], legs_hdvs[5], legs_hdvs[6], legs_hdvs[7], legs_hdvs[9]])

growth_hdv = bitHDV()
skin_protection = bitHDV()
limbs = bitHDV()

for i in 1:2
    feature_hdv = replace!(feature_hdv, feature_hdv[i] => bitbind(feature_hdv[i], skin_protection))
end

for i in 3:4
    feature_hdv = replace!(feature_hdv, feature_hdv[i] => bitbind(feature_hdv[i], growth_hdv))
end

for i in 12:13;16:21
    feature_hdv = replace!(feature_hdv, feature_hdv[i] => bitbind(feature_hdv[i], limbs))
end

hdv = BitVector[]
for i in 1:101
    v = BitVector[]
    for j in 2:16;18:23
        if data[!, j][i] == 1
            if j < 18
                push!(v, feature_hdv[j-1])
            end
            if j > 17
                push!(v, feature_hdv[j-2])
            end
        end
    end
    x = bitadd(v...)
    push!(hdv, x)
end

data[!, :species_hdv] = hdv

function grouper(group::Int)
    v = BitVector[]
    for i in 1:length(indices[group])
        push!(v, data.species_hdv[indices[group][i]])
    end
    x = bitadd(v...)
    return x
end

mammal_hdv = grouper(1)
bird_hdv = grouper(2)
reptile_hdv = grouper(3)
fish_hdv = grouper(4)
amphibian_hdv = grouper(5)
bug_hdv = grouper(6)
invertebrate_hdv = grouper(7)
\end{minted}
\subsection{Data analysis}
\begin{minted}[breaklines]{Julia}
list_group_hdvs = [mammal_hdv, bird_hdv, reptile_hdv, fish_hdv, amphibian_hdv, bug_hdv, invertebrate_hdv]
matrix_groups = permutedims(hcat(list_group_hdvs...))
\end{minted}
\textbf{\mintinline{Julia}{data.species_hdv[94]} refers to the 94th animal in the dataset, in this case that is vampire.}
\begin{minted}[breaklines]{Julia}
println("Similarity to mammal = ",string(hamming(data.species_hdv[94], mammal_hdv)))
println("Similarity to bird =",string(hamming(data.species_hdv[94], bird_hdv)))
println("Similarity to reptile =",string(hamming(data.species_hdv[94], reptile_hdv)))
println("Similarity to amphibian =",string(hamming(data.species_hdv[94], amphibian_hdv)))
println("Similarity to bug =",string(hamming(data.species_hdv[94], bug_hdv)))
println("Similarity to fish =",string(hamming(data.species_hdv[94], fish_hdv)))
println("Similarity to invertebrate =",string(hamming(data.species_hdv[94], invertebrate_hdv)))

axolotl_features = [0, 0, 1, 0, 0, 1, 0, 1, 1, 1, 0, 0, 1, 0, 0, 0, 0, 1, 0, 0, 0]
v = BitVector[]
    for j in 1:16
        if axolotl_features[j] == 1
            push!(v, feature_hdv[j])
        end
    end
    axolotl_hdv = bitadd(v...)


using MultivariateStats
M = fit(PCA, matrix_groups ; maxoutdim=2)
proj = projection(M)

using Plots
function plotter()
    colors = [:green, :orange, :blue, :red, :yellow, :purple, :black]
    labels = ["mammal", "bird", "reptile", "fish", "amphibian", "bug", "invertebrate"]
    fig = Plots.plot()

    for i in 1:7
        scatter!(fig, (proj[i, 1], proj[i, 2]), label=labels[i], mc = colors[i])
    end
    return fig
end
plotter()

matrix_species = permutedims(hcat(hdv...))
S = fit(PCA, matrix_species ; maxoutdim=2)
projS = projection(S)

using Plots
function plotter()
    colors = [:green, :orange, :blue, :red, :yellow, :purple, :black]
    labels = ["mammal", "bird", "reptile", "fish", "amphibian", "bug", "invertebrate"]
    fig = Plots.plot()

    for i in 1:7
        scatter!(fig, projS[indices[i], 1], projS[indices[i], 2], mc = colors[i], label=labels[i])
    end
    return fig
end
plotter()
\end{minted}
\subsection{Animal classifier}
\begin{minted}[breaklines]{Julia}
function predict(animal)
    y = [hamming(mammal_hdv, animal), hamming(bird_hdv, animal), hamming(reptile_hdv, animal), hamming(fish_hdv, animal), hamming(amphibian_hdv, animal), hamming(bug_hdv, animal), hamming(invertebrate_hdv, animal)]
    return findmin(y)[2]
end
using StatsBase
pred = [predict(i) for i in data.species_hdv]
mean(classls .== pred)
\end{minted}
\section{Anticancer protein example}
\subsection{Data preprocessing}
\begin{minted}[breaklines]{Julia}
using DataFrames, CSV
data = CSV.read("ProtExdata/ACPs_Breast_cancer.csv", DataFrame)

unique(data.class)
class_num = [i == "very active" ? 1 : i == "mod. active" ? 2 : i == "inactive - exp" ? 3 : 4 for i in data.class]
data[!, :class_num] = class_num
\end{minted}
\subsection{Assigning HDVs to features}
\begin{minted}[breaklines]{Julia}
AA_list = ['A', 'R', 'N', 'D', 'C', 'Q', 'E', 'G', 'H', 'I', 'L', 'K', 'M', 'F', 'P', 'S', 'T', 'W', 'Y', 'V', 'O', 'U', 'B', 'J', 'Z', 'X']
AA_hdv = [bitHDV() for i in AA_list]
AA_dict = Dict(zip(AA_list, AA_hdv))

trimer_hdvs = Dict(aa1 * aa2 * aa3 => 
bitbind(AA_dict[aa1], bitperm(AA_dict[aa2]), bitperm(AA_dict[aa3], 2)) 
for aa1 in AA_list for aa2 in AA_list for aa3 in AA_list)

function embedder(sequence)
    l = [trimer_hdvs[sequence[i:i+2]] for i in 1:length(sequence)-2]
    v = bitadd(hcat(l)...)
    return v
end
l = BitVector[]
for i in data.sequence
    push!(l, embedder(i))
end
data[!, :hdv] = l

active_hdv = bitadd(hcat([i for i in data[data.class_num .== 1, :hdv]])...)
modactive_hdv = bitadd(hcat([i for i in data[data.class_num .== 2, :hdv]])...)
notactive_exp_hdv = bitadd(hcat([i for i in data[data.class_num .== 3, :hdv]])...)
notactive_virt_hdv = bitadd(hcat([i for i in data[data.class_num .== 4, :hdv]])...)
\end{minted}
\subsection{Data analysis}
\textbf{Print the similarity of the HDV of sequence "KWKLFKKILKFLHLAKKF" to every class HDV}
\begin{minted}[breaklines]{Julia}
println(hamming(data[data.sequence .== , :hdv]..., active_hdv))
println(hamming(data[data.sequence .== "KWKLFKKILKFLHLAKKF", :hdv]..., modactive_hdv))
println(hamming(data[data.sequence .== "KWKLFKKILKFLHLAKKF", :hdv]..., notactive_exp_hdv))
println(hamming(data[data.sequence .== "KWKLFKKILKFLHLAKKF", :hdv]..., notactive_virt_hdv))
\end{minted}
\begin{minted}[breaklines]{Julia}
using MultivariateStats

matrix_all = permutedims(hcat(data.hdv...))
S = fit(PCA, matrix_all; maxoutdim=2)
projS = projection(S)

indices = [[i for i in 1:nrow(data) if data.class_num[i] == j] for j in 1:4]

using Plots
function plotter()
    colors = [:green, :black, :blue, :red]
    labels = ["active", "mod. active", "inactive (exp)", "inactive (virt)"]
    fig = Plots.plot()

    for i in 1:4
        scatter!(fig, (projS[indices[i], 1], projS[indices[i], 2]), label=labels[i], mc = colors[i])
    end
    return fig
end
plotter()
\end{minted}
\subsection{Classifier}
\begin{minted}[breaklines]{Julia}
n = nrow(data)

train = rand(n) .< 0.8
test = train = .! train

train_df = data[[i for i in 1:n if train[i] == 1], :]
test_df = data[[i for i in 1:n if test[i] == 1], :]

active_hdv_t = bitadd(hcat([i for i in train_df[train_df.class_num .== 1, :hdv]])...)
modactive_hdv_t = bitadd(hcat([i for i in train_df[train_df.class_num .== 2, :hdv]])...)
notactive_exp_hdv_t = bitadd(hcat([i for i in train_df[train_df.class_num .== 3, :hdv]])...)
notactive_virt_hdv_t = bitadd(hcat([i for i in train_df[train_df.class_num .== 4, :hdv]])...)

function predict(seq)
    y = [hamming(active_hdv_t, seq), hamming(modactive_hdv_t, seq), hamming(notactive_exp_hdv_t, seq), hamming(notactive_virt_hdv_t, seq)]
    return findmin(y)[2]
end

using StatsBase
pred = [predict(i) for i in test_df.hdv]
mean(test_df.class_num .== pred)
\end{minted}