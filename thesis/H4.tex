\chapter{PHALP case study: hyperdimensional protein sequence embedding}
\section{Phalp dataset}
To implement and evaluate hyperdimensional computing in a typical problem settings in computational biology, the potential of hyperdimensional computing will be evaluated on the PhaLP dataset~\cite{phalp} in this chapter. PhaLP is a comprehensive database currently comprising more than 17000 entries of phage lytic proteins including much of their information such as their type, domains and tertiary structures. Phage lytic proteins are used by bacteriophages to infect bacterial cells. To cross the bacterial cell walls, phages use two different types of phage lytic proteins: virion-associated lysins (VALs) and endolysins. Phage lytic proteins also comprise one or more functional domains categorized into two classes: enzymatically active domains (EADs) and cell wall binding domains (CBDs).

The escalating global antibiotic resistance crisis has necessitated the development of alternative strategies to combat bacterial infections. One such promising alternative is enzybiotics, a class of enzyme-based antibiotics derived from phage lytic proteins. Phage lytic proteins are produced by bacteriophages during their lytic replication cycle and are responsible for breaking down the bacterial cell wall. As these proteins exhibit a high level of diversity, it is critical to make well-informed selections during the early stages of research and development. In response to this need, the study introduced here presents PhaLP~\cite{phalp}, a comprehensive, automatically updated, and easily accessible database containing more than 17,000 phage lytic proteins. PhaLP aims to serve as a portal for researchers, allowing them to access all relevant information about the current diversity of phage lytic proteins through user-friendly search engines. This database is specifically designed to facilitate the development and application of enzybiotics by providing a wealth of data on protein architecture, evolution, and bacterial hosts corresponding to the phages. PhaLP not only serves as a valuable starting point for the broad community of enzybiotic researchers but also offers continually improving evolutionary insights that can act as a natural inspiration for protein engineers. By enabling researchers to make well-considered selections of phage lytic proteins during the early stages of their projects, PhaLP plays a significant role in the development of highly effective, narrow-spectrum antibiotics. These enzybiotics have the potential to revolutionize the field of antibacterial agents, offering a much-needed response to the alarming threat of antibiotic resistance that plagues healthcare systems worldwide.

To fully utilize the rich content of PhaLP, the researchers conducted a series of analyses at three levels to gain insights into the host-specific evolution of phage lytic proteins. First, they provided an overview of the modular diversity of these proteins. This was followed by the adoption of data mining and interpretable machine learning approaches to reveal host-specific design rules for domain architectures in endolysins. Lastly, the evolution of phage lytic proteins at the protein sequence level was explored, uncovering host-specific clusters.

In this chapter, we will explore the authors' experiment on protein classification based on sequence data and evaluates the potential of hyperdimensional computing in protein language modeling. The primary objective is to better understand the role of hyperdimensional computing in protein classification and to elucidate its potential advantages over conventional machine learning techniques.

\section{Type classifcation}
The developers of PhaLP aimed to classify protein sequences based on their type, with a focus on two types of phage lytic proteins: virion-associated lysins (VALs) and endolysins. Both of these proteins play crucial roles in the lytic replication cycle of bacteriophages, as they help the viruses breach the bacterial cell wall. VALs are an integral component of the viral particle, and their primary function is to create a small pore in the peptidoglycan layers of the bacterial cell wall at the infection site. This process allows the bacteriophage to gain access to the interior of the bacterial cell, initiating the lytic replication cycle. On the other hand, endolysins are produced within the infected bacterial cell and act towards the end of the replication cycle. These enzymes degrade the peptidoglycan layer of the bacterial cell wall, leading to cell lysis and the release of new viral particles.

Only a fraction of the database is manually annotated to include the protein's type because the amount of phage lytic proteins whose type is described in the literature is relatively small. The developers of PhaLP resorted to a machine learning approach for the classification of unannotated sequences. The authors embedded each protein sequence \textit{via} SeqVec~\cite{seqvec} and trained a random forest classifier with 100 estimators and balanced weights to classify the proteins whose types were unknown. For this case study, we attempted to simulate their experiments of classifying the proteins into the two types based on their sequence using several techniques based on hyperdimensional computing.

\section{Methods}
As of March 2023, the latest version of the PhaLP database,~\textit{v2021\_04}, has been used to test our models. This dataset consists of 17356 unique amino acid sequences of phage-lytic proteins.
\subsection*{Embedding of sequences into hyperdimensional vectors}
First, we used several sequence encoding techniques to embed the protein sequences in hyperdimensional space. In section~\ref{ssec: within the framework of hyperprotclas}, the bag-of-words (BoW) method of embedding sequences in hyperdimensional space has already been discussed. Here, a sequence of amino acids is considered to be a bag of k-mers. Within a k-mer, the amino acids (presented as randomly generated hyperdimensional vectors) are bound together with sequential information included. All possible k-mers are all then bundled together, the result is then a hyperdimensional vector representing the whole sequence. 

We also introduce a novel sequence embedding method in hypermensional computing. It is similar to the bag-of-words method in the sense that it bundles vectors of k-mers, but here, the k-mer's positional information will be encoded into the k-mer before bundling, similar to a convolutional layer. INSERT FIGURE 

These two sequence embedding methods have been applied to all sequences in the dataset using both both random hyperdimensional vectors and extended ESM-2 embeddings MAYBE MORE IN FUTURE. These were then visually assessed \textit{via} PCA.

\subsection*{Classification of hyperdimensional protein sequence embeddings}
